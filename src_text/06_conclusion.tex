\begin{conclusion}
    This work was focused on building a framework for the task of people tracking and soft-biometric data extraction. In the introductory chapter, we thoroughly described the assignment with sufficient motivation, but also with its challenges. In the following theoretical chapter, we have described the necessary theoretical background in detail for understanding of this thesis. In chapter \ref{related_work}, we briefly reviewed the existing solutions, and based on that some popular methods were implemented from scratch or customized from open-source repositories to meet the thesis goals. The implemented solution is then evaluated in the evaluation chapter, and further improvements are proposed. The result is a functioning people tracking and soft-biometrics extracting framework that can be deployed in a real-world application.
    
    According to the table \ref{table:tracking_results}, the proposed algorithms achieve {state-of-the-art} results in long-term people tracking. The best solution based on \gls{faster r-cnn} achieves only two identity switches on our dataset, which is an outstanding outcome. The quality of the detected boxes is also very high. Evaluated height and age estimators achieve exact results. However, in future work, it is necessary to evaluate more individuals.
    
    The proposed methods are designed in several variations to meet different trade-offs of accuracy versus computational expensiveness, and since the final design of the framework is fully modular, it can be easily configured to meet demands for various other scenarios. We put much effort into the application design as a whole so that it can be easily expanded and it can now be deployed in real-world applications. There are still some shortcomings in the framework, but they are described so that they can be further worked on and we hope that this work will be a useful as a starting point for other people interested in this topic. 

\end{conclusion}